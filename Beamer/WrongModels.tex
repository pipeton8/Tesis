% !BIB TS-program = biber
% !BIB program = biber
\documentclass[10pt, handout]{beamer}
%\documentclass[10pt]{beamer}

% Packages %
\usepackage[eng]{felipito}
%\usepackage{lmodern}
\usepackage[backend=biber,style=numeric-comp,sorting=none]{biblatex}
%\addbibresource{references.bib}
%\usepackage{coloremoji}

% Theme, color and font
\usetheme{Madrid}	
\usefonttheme{professionalfonts}
\beamertemplatenavigationsymbolsempty

% Epigraph
\setlength{\epigraphwidth}{0.89\textwidth}
\renewcommand{\textflush}{flushepinormal}
\renewcommand {\epigraphflush} {center}
\renewcommand {\sourceflush} {flushleft}

% Title and date
\title[]{Aggregate problems require aggregate solutions?}
\subtitle{When heterogeneity is expendable}
\author{Felipe Del Canto}
\institute[]{Pontificia Universidad Católica de Chile}
\date{\today}

% Graphics path
\graphicspath{{./Graphics/}}

% Documento %
\begin{document}

% Portada %
\frame[noframenumbering, plain]{\titlepage}

%: Outline
\begin{frame}[noframenumbering, plain]{Index}
	\tableofcontents
\end{frame}

%%%	Introduction		%%%
\section{Introduction}

%:	Outline
\begin{frame}[t]{Aggregation is an approximation}
	\begin{center}
		\epigraph{\scriptsize (...) all models are approximations. Essentially, all models are wrong, but some are useful. However, %
the approximate nature of the model must always be borne in mind.}{\scriptsize\textit{Empirical Model-Building and Response Surfaces (1987)} \\ \textsc{George Box and Norman Draper}}
	\end{center} 

\pause \vfill
	\begin{itemize}
		\item Aggregate models are, in particular, approximations. \pause \vfill
		\item The conditions to have exact aggregation are strict. \pause \vfill
		\item But the important measure of these models are their predictions. \pause \vfill
		\item Three questions: \vspace{.5ex}
			\begin{itemize}
				\item Is it possible to calibrate aggregate models to obtain better predictions? \vspace{1ex}
				\item How large can be the approximation error when aggregating? \vspace{1ex}
				\item How can we compare aggregate and disaggregate models?
			\end{itemize}
			\vfill
	\end{itemize}
\end{frame}

\begin{frame}{Aggregation ``$=$'' heterogeneity approximation}
	\vfill
	\begin{itemize} 
		\item Typically two types of aggregation are studied:
			\begin{itemize}
				\item Across consumers (representative agent models). \vspace{1ex}
				\item Across goods (i.e., two good models). 
			\end{itemize} \vfill \pause

		\item But aggregation is just approximating heterogeneity in preferences, income, geography, etc. \vfill \pause
			
		\item Aggregation ``$=$'' approximating distributions of heterogeneous variables:
			\begin{itemize}
				\item Distribution $F$ is approximated by a Dirac distribution $\delta_{x}$ at the point $x$. \vspace{1ex}
				\item $x$ is the choice of parameters of the aggregate model. 
			\end{itemize} \vfill

	\end{itemize}

\end{frame}

%:	Outline
\begin{frame}[noframenumbering, plain]
	\tableofcontents[currentsection]
\end{frame}

%%%	Aggregation across consumers		%%%
\section{Aggregation across consumers}

%:	Outline
\begin{frame}[noframenumbering, plain]
	\tableofcontents[currentsection]
\end{frame}

%%%	Aggregation in dynamical models		%%%
\section{Aggregation in dynamical models}

%:	Outline
\begin{frame}[noframenumbering, plain]
	\tableofcontents[currentsection]
\end{frame}

%%%	Representative agent and category goods		%%%
\section{Representative agent and category goods}

%:	Outline
\begin{frame}[noframenumbering, plain]
	\tableofcontents[currentsection]
\end{frame}

%%%	Aggregation across goods		%%%
\section{Aggregation across goods}

%:	Outline
\begin{frame}[noframenumbering, plain]
	\tableofcontents[currentsection]
\end{frame}

%%%	Final remarks		%%%
\section{Final remarks}


%:	Outline
\begin{frame}[noframenumbering, plain]
	\tableofcontents[currentsection]
\end{frame}


%:Contraportada
\begin{frame}[plain,noframenumbering]
  \titlepage
 \end{frame}



\end{document}

