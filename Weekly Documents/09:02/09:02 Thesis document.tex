\documentclass[english, a4paper,12pt]{article}

%:Preamble
\usepackage[bookmarks, colorlinks=true, allcolors=blue, pagebackref=true, hyperfootnotes=false]{hyperref}
\usepackage[eng]{felipito}
\usepackage[round]{natbib}

\graphicspath{{./Graphics/}}

\textheight = 690pt
\topmargin = -40pt
\textwidth = 500pt
\oddsidemargin = -22pt
\spacing{1.5}

%:Document
\begin{document}

%:	Title
\begin{center} \bf \large
	Hicksian aggregation season 2 \\ Felipe Del Canto, 2 de Septiembre de 2019
\end{center}

%:	Where have I seen this before?
\section{Where have I seen this before?}
Recall that in this problem, given $u : \R^{n+1} \to \R$, one wishes to find $f: \R^{n} \to \R$, $g: \R^{n} \to \R$ and $U: \R^{2} \to \R$ such that
	\begin{equation} \label{eq:probdisagg}
		\probop{\mathbf{x}(\mathbf{p}, q, y) 
			:= \argmax_{\mathbf{x}, z}}{u(\mathbf{x}, z)}
				{&	\mathbf{p} \mathbf{x} + qz = y.},
	\end{equation} 

and
	\begin{equation} \label{eq:probagg}
		\probop{X(f(\mathbf{p}), q, y) := \argmax_{X, z}}{U(X,z)}
										{&	PX + qz = y}
	\end{equation}

satisfy
	$$X(f(\mathbf{p}), q, y) = g\big(\mathbf{x}(\mathbf{p}, q, y)\big).$$
 
Typically, these functions do not exist except in very restrictive cases. Thus, in the general case the equality will not hold and consequently the approximation error could be directly measured by
	$$\clx{D}(\mathbf{p}, q,y) := \Big|X(f(\mathbf{p}), q, y) - g\big(\mathbf{x}(\mathbf{p}, q, y)\big)\Big|$$

For general functions $u$ and $U$ there is no algebraic expression of the demands $X$ and $\mathbf{x}$. Hence, in our last meeting we discussed a method to compute this difference based on a Taylor approximation of both demand functions. In order to obtain the derivatives one could implicitly differentiate the lagrangians of problems \eqref{eq:probdisagg} and \eqref{eq:probagg}. To put in practice this plan in a particular example one wishes to choose some ``nice'' utility functions $u$ and $U$ and aggregators $f$ and $g$ to see how these calculations work out. Sadly, the common Cobb-Douglas utility function is of no interest in this setting. Indeed, if $\alpha \in (0,1)^{n}$ with $|\alpha|_{1} < 1$, then
	$$\hat{u}_{\alpha}(\mathbf{x}, z) = \underbrace{\left(\prod_{j=1}^{n} x_{j}^{\alpha_{j}}\right)}_{:= v(\mathbf{x})}z^{1-|\alpha|_{1}},$$
and thus
	$$\hat{u}_{\alpha}(\mathbf{x},z) = \tilde{u}(v(\mathbf{x}), z),$$
with
	$$\tilde{u}(v,z) = vz^{1-|\alpha|_{1}}.$$
Since $v: \R^{n} \to \R$ is also an homothetic function, then functional separability holds and good aggregation is possible by choosing the proper aggregators $f$ and $g$. Consequently, in order to find an interesting example we must choose some function $u$ that does not allow $z$ to be ``independent'' of the goos in $x$. A simple tweak of the function $u_{\alpha}$ does this trick and gives some interesting results. Let $\alpha$ be as above and define
	\begin{equation} \label{eq:sumCD}
		u_{\alpha}(\mathbf{x},z) = \sum_{j=1}^{n} x_{j}^{\alpha_{j}}z^{1-\alpha_{j}}.
	\end{equation}

Observe that if all $\alpha_{j}$ are different, then it is not possible to factor $z$ in the above expression, solving the problem. Indeed, as \cite{VarianBook} points out, functional separability is present only if the preference relation $\succ_{u}$ represented by the utility function satisfies:
	\begin{equation} \label{eq:propfuncsep}
		(\mathbf{x}, z) \succ (\mathbf{x'}, z) 
			\Longleftrightarrow (\mathbf{x}, z') \succ (\mathbf{x'}, z') \qquad \paratodo z, z', \mathbf{x}, \mathbf{x'}
	\end{equation}
It is straightforward to see that for the preference relation represented by the function $u_{\alpha}$ defined in \eqref{eq:sumCD}, property \eqref{eq:propfuncsep} does not hold unless all $\alpha_{j}$ are equal. 

Observe that when $\alpha_{j} = \alpha_{0}$ for every $j$, then
	$$u_{\alpha_{0}}(\mathbf{x},z) = \left(\sum_{j=1}^{n} x_{j}^{\alpha_{0}}\right)z^{1-\alpha_{0}}.$$
The parenthesis in this last expression suggest a particular form of aggregation
	$$g(\mathbf{x}) := \left(\sum_{j=1}^{n} x_{j}^{\alpha_{0}}\right)^{1/\alpha_{0}}.$$
	
This function $g$ is also present in the context of international trade. In \cite{TradeFlows} the authors develop a model where $g$ is precisely the utility function of a country. They also suggest a certain price index that could also be useful in this setting but I have not investigated it yet. Given this function $g$, the aggregate function $U$ should be defined as follows:
	$$U_{\alpha_{0}}(X, z) = X^{\alpha_{0}}z^{1-\alpha_{0}},$$
which is the usual Cobb-Douglas utility function. In the general case, when the $\alpha_{j}$ are not equal, we could define $U$ as follows:
	$$U_{\alpha}(X,z) = X^{\overline{\alpha}}z^{1-\overline{\alpha}},$$
where $\overline{\alpha} = \frac{1}{n}\sum_{j=1}^{n} \alpha_{j}$. This definition has the property that if $\alpha_{j} = \alpha_{0}$ for every $j$, then $\overline{\alpha} = \alpha_{0}$.

The only problem with the function $u_{\alpha}$ is the following. The first order conditions of problem \eqref{eq:probdisagg} are
	\begin{align}
		\alpha_{j}x_{j}^{\alpha_{j}}z^{1-\alpha_{j}}	&=	\lambda_{d}p_{j}x_{j},	\qquad j= 1,\ldots,n	\label{eq:CPOd1}	\\
		\sum_{j=1}^{n} x_{j}^{\alpha_{j}}z^{1-\alpha_{j}} - \sum_{j=1}^{n}\alpha_{j}x_{j}^{\alpha_{j}}z^{1-\alpha_{j}}
											&=	\lambda_{d}qz							\label{eq:CPOd2}
	\end{align}
Summing \eqref{eq:CPOd1} over $j$ and replacing in \eqref{eq:CPOd2} we have
	$$\sum_{j=1}^{n} x_{j}^{\alpha_{j}}z^{1-\alpha_{j}} - \lambda_{d} \sum_{j=1}^{n} p_{j}x_{j} = \lambda_{d}qz,$$
and thus
	$$\lambda_{d} = \frac{u(x,z)}{y}.$$

This means we do not have an analytic expression for the demand $\mathbf{x}(\mathbf{p}, q,y)$ but computing the derivatives should be possible by implicit differentiation of the lagrangian. The derivatives of $X$ can be easily computed since
	$$X(f(\mathbf{p}), q, y) = \overline{\alpha}\frac{y}{f(\mathbf{p})}$$

%: 	Bibliography
\bibliographystyle{abbrvnat}
\bibliography{references}

\end{document}